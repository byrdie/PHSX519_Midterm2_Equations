% --------------------------------------------------------------
% This is all preamble stuff that you don't have to worry about.
% Head down to where it says "Start here"
% --------------------------------------------------------------
 
\documentclass[10pt]{article}
 
\usepackage[margin=.60in]{geometry} 
\usepackage{amsmath,amsthm,amssymb, mathtools}
\usepackage{multicol}
\usepackage[subnum]{cases}
\usepackage{relsize}
\usepackage[makeroom]{cancel}
\usepackage[english]{babel}
\usepackage{graphicx}
\usepackage{calligra}
\usepackage[normalem]{ulem}
\usepackage{caption}
\usepackage{subcaption}
\usepackage{fancyhdr}

\DeclareMathAlphabet{\mathcalligra}{T1}{calligra}{m}{n} 
\DeclareFontShape{T1}{calligra}{m}{n}{<->s*[2.2]callig15}{}


% Makes '\sr' make a script r
\newcommand{\sr}{\ensuremath{\mathcalligra{r}}}
 
\newcommand{\N}{\mathbb{N}}
\newcommand{\Z}{\mathbb{Z}}
\newcommand{\ihat}{\boldsymbol{\hat{\textbf{\i}}}}
\newcommand{\jhat}{\boldsymbol{\hat{\textbf{\j}}}}
\newcommand{\khat}{\boldsymbol{\hat{\textbf{k}}}}
\newcommand{\rhat}{\boldsymbol{\hat{\textbf{r}}}}
\newcommand{\srhat}{\boldsymbol{\hat{\textbf{\sr}}}}
\newcommand{\xhat}{\boldsymbol{\hat{\textbf{x}}}}
\newcommand{\yhat}{\boldsymbol{\hat{\textbf{y}}}}
\newcommand{\zhat}{\boldsymbol{\hat{\textbf{z}}}}
\newcommand{\nhat}{\boldsymbol{\hat{\textbf{n}}}}
\newcommand{\phihat}{\boldsymbol{\hat{\textbf{$\phi$}}}}

\newcommand{\vect}[1]{\boldsymbol{\mathbf{#1}}}
\newcommand{\vc}[1]{\mathbf{#1}}
\newcommand{\fracl}[2]{\mathlarger{\frac{#1}{#2}}}
\newcommand{\dd}{\, \mathrm{d}}
\newcommand{\eo}{\epsilon_0}
\newcommand{\mo}{\mu_\circ}
\newcommand{\tder}[2]{\frac{\dd #1}{\dd #2}}
\newcommand{\pder}[2]{\frac{\partial #1}{\partial #2}}
\newcommand{\dtder}[2]{\frac{\dd^2 #1}{\dd #2^2}}
\newcommand{\dpder}[2]{\frac{\partial^2 #1}{\partial #2^2}}
\newcommand{\intas}{ \int_{-\infty}^\infty}
\newcommand{\wt}[1]{\widetilde{#1}}
\newcommand{\ev}[1]{\left\langle #1 \right\rangle}
\newcommand{\ce}{\wt{\vect{E}}}
\newcommand{\cb}{\wt{\vect{B}}}
\newcommand{\K}{\frac{1}{4 \pi \eo}}
 
\newenvironment{theorem}[2][Theorem]{\begin{trivlist}
\item[\hskip \labelsep {\bfseries #1}\hskip \labelsep {\bfseries #2.}]}{\end{trivlist}}
\newenvironment{lemma}[2][Lemma]{\begin{trivlist}
\item[\hskip \labelsep {\bfseries #1}\hskip \labelsep {\bfseries #2.}]}{\end{trivlist}}
\newenvironment{exercise}[2][Exercise]{\begin{trivlist}
\item[\hskip \labelsep {\bfseries #1}\hskip \labelsep {\bfseries #2.}]}{\end{trivlist}}
\newenvironment{problem}[2][Problem]{\begin{trivlist}
\item[\hskip \labelsep {\bfseries #1}\hskip \labelsep {\bfseries #2.}]}{\end{trivlist}}
\newenvironment{question}[2][Question]{\begin{trivlist}
\item[\hskip \labelsep {\bfseries #1}\hskip \labelsep {\bfseries #2.}]}{\end{trivlist}}
\newenvironment{corollary}[2][Corollary]{\begin{trivlist}
\item[\hskip \labelsep {\bfseries #1}\hskip \labelsep {\bfseries #2.}]}{\end{trivlist}}


\newenvironment{Figure}
  {\par\medskip\noindent\minipage{\linewidth}}
  {\endminipage\par\medskip}

\pagestyle{fancy}
\lhead{Midterm 2 Equations}
\chead{PHSX519 Electromagnetic Theory I}
\rhead{Roy Smart}

 
\begin{document}
\begin{multicols}{2}
	\begin{align}	
		&q = \int \rho(\vect{x}') \; d^3x'	\tag*{Monopole (4.4)} \\
		&\vect{p} = \int \vect{x}' \rho(\vect{x}') \; d^3x'	\tag*{Dipole (4.8)} \\
		&Q_{ij} = \int (3 x_i' x_j'- r' \delta_{ij})\rho(\vect{x}') d^3x'		\tag*{Quadrupole (4.9)} \\
		&\Phi(\vect{x}) = \K \left[ \frac{q}{r} + \frac{ \vect{p} \cdot \vect{x}}{r^3} + \frac{1}{2} \sum_{i,j} Q_{ij} \frac{x_i x_j}{r^5} + ... \right] \tag*{Multipole Expansion (4.10)} \\
		& E_r=\frac{2 p \cos \theta}{4 \pi \eo}, \quad E_\theta=\frac{2 p \sin \theta}{4 \pi \eo} \tag*{Dipole in $\zhat$ (4.12)} \\
		& \vect{E}(\vect{x}) = \frac{3 \vect{n}(\vect{p} \cdot \vect{n}) - \vect{p}}{4 \pi \eo |\vect{x} - \vect{x}_0|^3} \tag*{$\vect{E}$-field due to dipole $\vect{p}$ (4.13)} \\
		& W = \int \rho(\vect{x}) \Phi(\vect{x})\; d^3x \tag*{Charge in \textit{external} field (4.21)} \\
		& W = q \Phi(0) - \vect{p} \cdot \vect{E}(0) - \frac{1}{6} \sum_i \sum_j Q_{ij} \pder{E_j}{x_i}(0) + ... \tag*{Energy multipole expansion (4.24)} \\
		& \vect{P}(\vect{x}) = \sum_i N_i \ev{\vect{p}_i} \tag*{Electric polarization (4.28)} \\
		& \vect{D} = \eo \vect{E} + \vect{P} \tag*{Electric displacement(4.34)} \\
		& \vect{P} = \eo \chi_e \vect{E} \tag*{Induced polarization (4.36)} \\
		& \vect{D} = \epsilon \vect{E} \tag*{Electric displacement (4.37)} \\
		& \epsilon = \eo (1 + \chi_e) \tag*{Electric permittivity (4.38)} \\
		& \begin{cases}
			(\vect{D}_2 - \vect{D}_1) \cdot \vect{n}_{21} = \sigma \\
			(\vect{E}_2 - \vect{E}_1 \times \vect{n}_{21}) = 0 \\
		\end{cases}	\tag*{Boundary conditions (4.40)} \\
		\begin{split}
			&\Phi_{\text{in}} = - \left( \frac{3}{\epsilon/\eo + 2} \right) E_0 r \cos \theta \\ 
			&\Phi_{\text{out}} = - E_0 r \cos \theta + \left( \frac{\epsilon/\eo - 1}{\epsilon/\eo + 2} \right) E_0 \frac{a^3}{r^2} \cos \theta 
		\end{split} \tag*{Dielectric sphere in uniform $\vect{E}$-field (4.54)} \\
		& W = \frac{1}{2} \int \vect{E} \cdot \vect{D} \; d^3x \tag*{Electrostatic Energy (4.89)} \\
		& W = - \frac{1}{2} \int_{V_1} \vect{P} \cdot \vect{E}_0 \; d^3 x \tag*{Dielectric placed in $\vect{E}_0$ (4.93)} \\
		& \vect{N} = \vect{\mu} \times \vect{B} \tag*{Torque on magnetic dipole moment (5.1)} \\
		& \nabla \cdot \vect{J} = 0 \tag*{Condition of magnetostatics (5.3)} \\
		& d \vect{B} = k I \frac{d \vect{l} \times \vect{x}}{|\vect{x}|^3} \tag*{Biot-Savart Law (5.4)} \\
		& \vect{F} = \int \vect{J}(\vect{x}) \times \vect{B}(\vect{x}) \; d^3 x \tag*{Force on current dist. (5.12)} \\
		& \vect{N} = \int \vect{x} \times (\vect{J} \times \vect{B}) \; d^3 x \tag*{Torque on current dist. (5.13)} \\
		& \oint_C \vect{B} \cdot d \vect{l} = \mu_0 I \tag*{Amp\`eres law (5.25)}\\
		& \vect{B}(\vect{x}) = \nabla \times \vect{A}(\vect{x}) \tag*{Magnetic vector potential (5.27)} \\
		& \vect{A}(\vect{x}) = \frac{\mu_0}{4 \pi} \int \frac{\vect{J}(\vect{x}')}{|\vect{x}-\vect{x}'|} d^3 x' \tag*{MVP from current dist. (5.32)} \\
	\end{align}
\end{multicols}
% --------------------------------------------------------------
%     You don't have to mess with anything below this line.
% --------------------------------------------------------------
 
\end{document}
